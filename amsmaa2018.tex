\documentclass[12pt]{article}

\usepackage[letterpaper, margin=1.2in]{geometry}

\usepackage{amsmath,amsthm,fancyhdr}
\usepackage[charter]{mathdesign}

\theoremstyle{definition}
\newtheorem{problem}{Problem}

\pagestyle{fancy}
\lhead{AMS/MAA 2018 Joint Meetings}
\rhead{MAA Session on Mathematics and Sports}
\cfoot{\thepage}

\begin{document}
\begin{center}
\textbf{Measuring Umpire Consistency} \\
\textsc{David J. Hunter}
\end{center}

The availability of pitch-tracking data has led to increased scrutiny of Major League Baseball umpires. While many studies have attempted to rate umpires based on their conformity to the rule book strike zone, players and managers tend to accept deviations from this zone, provided that umpires establish consistent zones within a game. In this talk, I will apply techniques from computational geometry to create a new metric for measuring the consistency of an umpire's ball and strike calls. This metric yields a consistency ranking of MLB umpires using pitch-tracking data on all ball and strike calls made during the 2017 MLB regular season. I will also discuss generalizations of this metric and applications beyond baseball.

\end{document}
