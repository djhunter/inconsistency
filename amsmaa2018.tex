\documentclass[12pt]{article}

\usepackage[letterpaper, margin=1.2in]{geometry}

\usepackage{amsmath,amsthm,fancyhdr}
\usepackage[charter]{mathdesign}

\theoremstyle{definition}
\newtheorem{problem}{Problem}

\pagestyle{fancy}
\lhead{AMS/MAA 2018 Joint Meetings}
\rhead{MAA Session on Mathematics and Sports}
\cfoot{\thepage}

\begin{document}
\begin{center}
 \textbf{Measuring Umpire Consistency} \\
 \textsc{David J. Hunter}
\end{center}

The availability of pitch-tracking data has led to increased scrutiny of Major League Baseball umpires. While many studies have attempted to rate umpires based on their conformity to the rule book strike zone, players and managers tend to accept deviations from this zone, provided that umpires establish consistent zones within a game. Using tools from computational geometry, we construct a new metric to assess the consistency of an umpire's ball and strike calls over the course of a game. This metric yields a consistency ranking of MLB umpires using pitch-tracking data on all ball and strike calls made during the 2017 MLB regular season. It can also be used to measure consistency of umpires under various conditions (e.g., for different pitch types, stadiums, and catchers). This talk will address the motivation for this metric, explore some of its applications and generalizations, and discuss its statistical properties.

\end{document}
